\documentclass{article}
\usepackage{amsmath}
\usepackage{amsfonts}

\begin{document}

\title{Homework Assignment}
\author{Yaghoub Shahmari - 98100883}
\date{}
\maketitle

\section*{Problem 7.2:}

\subsection*{a.}
$$ p_k = \frac{N-1}{N}\delta(k-1) + \frac{1}{N}\delta(k-(N-1)) $$

\subsection*{b.}
$$ q_k = \frac{\delta (k-1) + \delta (k-(N-1))}{2} $$

\subsection*{c.}
$$ e_{jk} = \frac{\delta_{j,1}\delta_{k,N-1} + \delta_{j,N-1}\delta_{k,1}}{2} $$

$$ r = \sum_{j,k}\frac{jk(e_{jk}-q_j q_k)}{\sigma^2} $$

$$ \sigma^2 = \sum_k k^2q_k - \left(\sum_k kq_k\right)^2 $$

$$ \Rightarrow r = \frac{(N-1)-\left(\frac{N^2}{4}\right)}{\frac{1+(N-1)^2}{2}-\left(\frac{N^2}{4}\right)} = -1 $$

\subsection*{d.}
All of the nodes with small degree are connected to the nodes with large degree. Therefore, the network is disassortative.

\section*{Problem 7.4:}

\subsection*{a.}
In the Erdős-Rényi $G(N,L)$ model, the probability that there is a link between nodes $i$ and $j$ is given by:

$$ e_{ij} = \frac{L}{\binom{N}{2}} $$

Where $\binom{N}{2}$ is the total number of possible links.

The conditional probability that there is a link between $i$ and $j$ given there is a link between $l$ and $s$ is:

$$ e_{ij}|e_{ls} = \frac{L-1}{\binom{N}{2}-1} $$

\subsection*{b.}
For small networks, the ratio of these probabilities is:

$$ \frac{e_{ij}|e_{ls}}{e_{ij}} = \frac{\frac{L-1}{\binom{N}{2}-1}}{\frac{L}{\binom{N}{2}}} \approx 1 $$

For large networks (as $N \to \infty$), this ratio approaches 1.

\subsection*{c.}
In the Erdős-Rényi $G(N,p)$ model, the probabilities are:

$$ e_{ij} = p $$

$$ e_{ij}|e_{ls} = p $$

These probabilities are independent.

The ratio is always 1, regardless of network size.

Implications for small networks:

\begin{itemize}
    \item Using the $G(N,L)$ model introduces correlations between link existence probabilities that are not present in the $G(N,p)$ model.
    \item For model validation or simulation of small real-world networks, the $G(N,p)$ model may not capture these correlations appropriately.
    \item The ratio of conditional to unconditional probabilities being $\approx 1$ in $G(N,L)$ for small $N$ indicates significant correlations.
    \item The $G(N,L)$ model may better represent the linking process in small real networks.
\end{itemize}

So for modeling and analyzing small networks, using the $G(N,L)$ over $G(N,p)$ model is likely more appropriate to account for degree correlations not present in $G(N,p)$.

\end{document}
